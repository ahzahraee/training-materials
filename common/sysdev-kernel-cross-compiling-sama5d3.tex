\section{Flashing the kernel and DTB in NAND flash}

In order to let the kernel boot on the board autonomously, we can
flash the kernel image and DTB in the NAND flash available on the
Xplained board.

After storing the first stage bootloader, U-boot and its environment
variables, we will keep special areas in NAND flash for the DTB
and Linux kernel images:

\begin{center}
  \includegraphics[width=\textwidth]{labs/sysdev-kernel-cross-compiling/flash-map.pdf}
\end{center}

So, let's start by erasing the corresponding 128 KiB of NAND flash
for the DTB:

% Verbatim doesn't seem to work in this \ifdefstring environment

\begin{ubootinput}
=> nand erase 0x180000 0x20000} %\rmfamily({\em NAND-offset size})%
\end{ubootinput}

Then, let's erase the 5 MiB of NAND flash for the kernel image:

% Verbatim doesn't seem to work in this \ifdefstring environment
\ubootcmd{=> nand erase 0x1a0000 0x500000}

Then, copy the DTB and kernel binaries from TFTP into memory, using the
same addresses as before.

Then, flash the DTB and kernel binaries:

% Verbatim doesn't seem to work in this \ifdefstring environment
\begin{ubootinput}
=> nand write 0x22000000 0x180000 0x20000} %\rmfamily({\em RAM-addr NAND-offset size})%
=> nand write 0x21000000 0x1a0000 0x500000
\end{ubootinput}

Power your board off and on, to clear RAM contents. We should now be
able to load the DTB and kernel image from NAND and boot with:

% Verbatim doesn't seem to work in this \ifdefstring environment
\begin{ubootinput}
=> nand read 0x22000000 0x180000 0x20000} %\rmfamily({\em RAM-addr offset size})%
=> nand read 0x21000000 0x1a0000 0x500000
=> bootz 0x21000000 - 0x22000000
\end{ubootinput}

Write a U-Boot script that automates the DTB + kernel download
and flashing procedure.

You are now ready to modify \code{bootcmd} to boot the board
from flash. But first, save the settings for booting from \code{tftp}:

% Verbatim doesn't seem to work in this \ifdefstring environment
\begin{ubootinput}
=> setenv bootcmdtftp ${bootcmd}
\end{ubootinput}

This will be useful to switch back to \code{tftp} booting mode
later in the labs.

Finally, using \code{editenv bootcmd},
adjust \code{bootcmd} so that the Xplained board starts
using the kernel in flash.

Now, reset the board to check that it boots
in the same way from NAND flash. Check that this is really your own version of
the kernel that's running\footnote{Look at the kernel log. You will find
the kernel version number as well as the date when it was compiled.
That's very useful to check that you're not loading an older version
of the kernel instead of the one that you've just compiled.}

