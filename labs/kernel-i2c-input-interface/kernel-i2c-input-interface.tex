\subchapter{Input interface}{Objective: make the I2C device available
to user space using the input subsystem.}

After this lab, you will be able to:

\begin{itemize}
\item Expose device events to user space through an input interface,
      using the kernel based polling API for input devices
      (kernel space perspective).
\item Handle registration and allocation failures in a clean
      way.
\item Get more familiar with the usage of the input interface
      (user space perspective).
\end{itemize}

\section{Add input event interface support to the kernel}

For this lab, you need to rebuild your kernel with static input
event interface (\kconfig{CONFIG_INPUT_EVDEV}) support. With the default
configuration, this feature is available as a module, which is less
convenient.

Update and reboot your kernel.

\section{Register an input interface}

The first thing to do is to add an input device to the system. Here are
the steps to do it:

\begin{itemize}
\item Declare a pointer to an \ksym{input_dev} structure in the
      \code{probe} routine. You can call it \code{input}.
      You can't use a global variable because your driver needs to be
      able to support multiple devices.
\item Allocate such a structure in the same function, using the
      \kfunc{devm_input_allocate_device} function.
\item Still in the \code{probe()} function, add the input device to
      the system by calling \kfunc{input_register_device};
\end{itemize}

At this stage, first make sure that your module compiles well (add
missing headers if needed).

\section{Handling probe failures}

In the code that you created, make sure that you handle failure
situations properly.

\begin{itemize}
\item Of course, test return values properly and log
      the causes of errors.
\item In our case, we only allocated resources with \code{devm_}
      functions. Thanks to this, in case of failure, all the
      corresponding allocations are automatically released
      before destroying the \ksym{device} structure for each
      device. This greatly simplifies our error management code!
\end{itemize}

\section{Implement the remove() function}

In this function, we need to unregister and release the resources allocated
and registered in the \code{probe()} routine.

Fortunately, in our case, there's nothing to do, as everything
was allocated with \code{devm_} functions. Even the unregistration
of the \ksym{input_dev} structure is automated. 

Recompile your module, and load it and remove it multiple times, to
make sure that everything is properly registered and automatically
unregistered.

\section{Add proper input device registration information}

We actually need to add more information to the \ksym{input} structure before
registering it. That's why we are getting the below warnings:

\begin{verbatim}
input: Unspecified device as /devices/virtual/input/input0
\end{verbatim}

Add the below lines of code (still before device registration, of
course):

\sourcecode{labs/kernel-i2c-input-interface/input-device-attributes.c}

Recompile and reload your driver. You should now see in the kernel log
that the \code{Unspecified device} type is replaced by
\code{Wii Nunchuk}.

\section{Implement a polling routine}

The nunchuk doesn't have interrupts to notify the I2C master that
its state has changed. Therefore, the only way to access device data
and detect changes is to regularly poll its registers.

So, it's time to implement a routine which will poll the nunchuk registers
at a regular interval.

Create a \code{nunchuk_poll()} function with the right prototype (find
it by looking at the definition of the \kfunc{input_setup_polling} function.)

In this function, you will have to read the nunchuk registers. However,
as you can see, the prototype of the \code{poll_fn()} routine doesn't
carry any information about the \ksym{i2c_client} structure you will
need to communicate with the device. That's normal as the input
subsystem is generic, and can't be bound to any specific bus.

This raises a very important aspect of the device model: the need to
keep pointers between {\em physical} devices (devices as handled by the
physical bus, I2C in our case) and {\em logical} devices (devices
handled by subsystems, like the input subsystem in our case).

This way, when the \code{remove()} routine is called, we can find out
which logical device to unregister (though that's not necessary in our
case as logical device unregistration is automatic). Conversely, when we
have an event on the logical side (such as running the polling
function), we can find out which I2C device this corresponds to,
to communicate with the hardware.

This need is typically implemented by creating a per device, {\em private} data
structure to manage our device and implement such pointers between
the physical and logical worlds.

Add the below global definition to your code:

\begin{verbatim}
struct nunchuk_dev {
        struct i2c_client *i2c_client;
};
\end{verbatim}

Now, in your \code{probe()} routine, declare an instance of
this structure:

\begin{verbatim}
struct nunchuk_dev *nunchuk;
\end{verbatim}

Then allocate one such instead for each new device:

\sourcecode{labs/kernel-i2c-input-interface/private-data-alloc.c}

Note that we haven't seen kernel memory allocator routines and flags
yet.

Also note that here there's no need to write an "out of memory"
message to the kernel log. That's already done by the memory subsystem.

Now implement the pointers that we need:

\sourcecode{labs/kernel-i2c-input-interface/device-pointers.c}

Make sure you add this code before registering the input device. You
don't want to enable a device with incomplete information or when it is
not completely initialized yet (there could be race conditions).

So, back to the \code{nunchuk_poll()} function, you will first need to
retrieve the I2C physical device from the \ksym{input_dev}
structure. That's where you will use your private \code{nunchuk}
structure.

Now that you have a handle on the I2C physical device, you can move the
code reading the nunchuk registers to this function. You can
remove the double reading of the device state, as the polling function
will make periodic reads anyway\footnote{During the move, you will have
to handle communication errors in a slightly different way, as the
\code{nunchuk_poll()} routine has a \code{void} type. When the function
reading registers fails, you can use a \code{return;} statement instead
of \code{return value;}}.

At the end of the polling routine, the last thing to do is post the events
and notify the \code{input} core. Assuming that \code{input} is the
name of the \code{input_dev} parameter of your polling routine:

\sourcecode{labs/kernel-i2c-input-interface/input-notification.c}

Now, back to the \code{probe()} function, the last thing to do
is to declare the new polling function (see the slides if you forgot
about the details) and specify a polling interval of 50 ms.

At this stage, also remove the debugging messages about the state
of the buttons. You will get that information from the input interface.

You can now make sure that your code compiles and loads successfully.

\section{Testing your input interface}

Testing an input device is easy with the \code{evtest} application
that is included in the root filesystem. Just run:

\begin{verbatim}
evtest
\end{verbatim}

The application will show you all the available input devices, and will let
you choose the one you are interested in (make sure you type a choice,
\code{0} by default, and do not just type \code{[Enter]}). You can also
type \code{evtest /dev/input/event0} right away.

Press the various buttons and see that the corresponding events are
reported by \code{evtest}.

\section{Going further}

Stopping here is sufficient, but if you complete your lab before the
others, you can try to achieve the below challenges (in any order):

\subsection{Supporting multiple devices}

Modify the driver and Device Tree to support two nunchuks at the same
time. You can borrow another nunchuk from the instructor or from a fellow
participant.

Making sure that your driver does indeed support multiple devices at the
same time is a good way to make sure it is implemented properly.

\subsection{Use the nunchuk as a joystick in an ascii game}

In this optional, challenge, you will extend the driver to expose
the joystick part of the nunchuk, i.e. x and y coordinates.

We will use the {\em nInvaders} game, which is already present in
your root filesystem.

\subsubsection{Connect through SSH}

{\em nInvaders} will not work very well over the serial port,
so you will need to log to your system through \code{ssh} in an
ordinary terminal:

\begin{verbatim}
ssh root@192.168.0.100
\end{verbatim}

The password for the {\em root} user is \code{root}.

You can already play the \code{nInvaders} game with
the keyboard!

Note: if you get the error \code{Error opening terminal: xterm-256color.}
when running \code{nInvaders}, issue first the
\code{export TERM=xterm} command.

\subsubsection{Recompile your kernel}

Recompile your kernel with support for the joystick interface
(\kconfig{CONFIG_INPUT_JOYDEV}).

Reboot to the new kernel.

\subsubsection{Extend your driver}

We are going to expose the joystick X and Y coordinates through
the input device.

Add the below code to the \code{probe} routine:
\sourcecode{labs/kernel-i2c-input-interface/declare-x-and-y.c}

See \kerneldochtml{input/input-programming} for details about
the \kfunc{input_set_abs_params} function.

For the joystick to be usable by the application, you will also
need to declare {\em classic} buttons:

\sourcecode{labs/kernel-i2c-input-interface/declare-classic-buttons.c}

The next thing to do is to retrieve and report the joystick X and Y
coordinates in the polling routine. This should be very straighforward.
You will just need to go back to the nunchuk datasheet to find out
which bytes contain the X and Y values.

\subsubsection{Time to play}

Recompile and reload your driver.

You can now directly play {\em nInvaders}, only with your nunchuk.
You'll quickly find how to move your ship, how to shoot and how
to pause the game.

Have fun!
