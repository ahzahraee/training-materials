\subchapter{Lab2: Advanced Yocto configuration}{Configure the build, customize the
	output images and use NFS}

During this lab, you will:
\begin{itemize}
  \item Customize the package selection
  \item Configure the build system
  \item Use the rootfs over NFS
\end{itemize}

\section{Set up the Ethernet communication and NFS on the board}

Later on, we will mount our root filesystem through the network using
NFS. We will use Ethernet over USB device and therefore will only need
the USB device cable that is already used to power up the board.

First we need to set the kernel boot arguments U-Boot will pass to the
Linux kernel at boot time:
\begin{verbatim}
setenv bootargs 'console=ttyS0,115200 root=/dev/nfs rw
  nfsroot=192.168.0.1:/nfs,nfsvers=3,tcp
  ip=192.168.0.100:::::usb0 g_ether.dev_addr=f8:dc:7a:00:00:02
  g_ether.host_addr=f8:dc:7a:00:00:01'
setenv bootcmd 'mmc dev 0; devnum=${mmcdev}; setenv devtype mmc;
  mmc rescan; run loadimage; run findfdt; run mmcloados'
saveenv
\end{verbatim}

If you later want to make changes to this setting, you can use:
\begin{verbatim}
editenv bootargs
\end{verbatim}

\section{Set up the Ethernet communication on the workstation}

To configure your network interface on the workstation side, we need
to know the name of the network interface connected to your board. You
won't be able to see the network interface corresponding to the
Ethernet over USB device connection yet, because it's only active when
the board turns it on, from U-Boot or from Linux. When this happens,
the network interface name will be \code{enx<macaddr>}. Given the
value we gave to \code{g_ether.host_addr}, it will therefore be
\code{enxf8dc7a000001}.

Then, instead of configuring the host IP address from Network
Manager's graphical interface, let's do it through its command line
interface, which is so much easier to use:

\begin{verbatim}
nmcli con add type ethernet ifname enxf8dc7a000001 ip4 192.168.0.1/24
\end{verbatim}

\section{Set up the NFS server on the workstation}

First install the NFS server on the training computer and create the root NFS
directory:
\begin{verbatim}
sudo apt install nfs-kernel-server
sudo mkdir -m 777 /nfs
\end{verbatim}

Then make sure this directory is used and exported by the NFS server by adding
\code{/nfs *(rw,sync,no_root_squash,subtree_check)} to the
\code{/etc/exports} file.

Finally restart the service:
\begin{verbatim}
sudo service nfs-kernel-server restart
\end{verbatim}

\section{Add a package to the rootfs image}

You can add packages to be built by editing the local configuration file
\code{$BUILDDIR/conf/local.conf}. The \code{IMAGE_INSTALL} variable controls the
packages included into the output image.

To illustrate this, add the Dropbear SSH server to the list of enabled
packages.

Tip: do not override the default enabled package list, but append the Dropbear
package instead.

\section{Boot with the updated rootfs}

First we need to put the rootfs under the NFS root directory so that it is
accessible by NFS clients. Simply uncompress the archived output image in the
previously created \code{/nfs} directory:
\begin{verbatim}
sudo tar xpf $BUILDDIR/tmp/deploy/images/beaglebone/\
  core-image-minimal-beaglebone.tar.xz -C /nfs
\end{verbatim}

Then boot the board.

The Dropbear SSH server was enabled a few steps before, and should now be
running as a service on the BeagleBone Black. You can test it by accessing the
board through SSH:
\begin{verbatim}
ssh root@192.168.0.100
\end{verbatim}

You should see the BeagleBone Black command line!

\section{Choose a package variant}

Dependencies of a given package are explicitly defined in its recipe.
Some packages may need a specific library or piece of software but
others only depend on a functionality. As an example, the kernel
dependency is described by \code{virtual/kernel}.

To see which kernel is used, dry-run BitBake:
\begin{verbatim}
bitbake -vn virtual/kernel
\end{verbatim}

In our case, we can see the \code{linux-ti-staging} provides the
\code{virtual/kernel} functionality:
\small
\begin{verbatim}
NOTE: selecting linux-ti-staging to satisfy virtual/kernel due to PREFERRED_PROVIDERS
\end{verbatim}
\normalsize

We can force Yocto to select another \code{kernel} by explicitly
defining which one to use in our local configuration. Try switching
from \code{linux-ti-staging} to \code{linux-dummy} only using the
local configuration.

Then check the previous step worked by dry-running again BitBake.
\begin{verbatim}
bitbake -vn virtual/kernel
\end{verbatim}

You can now rebuild the whole Yocto project, with \code{bitbake
core-image-minimal}

Tip: you need to define the more specific information here to be sure it is the
one used. The \code{MACHINE} variable can help here.

As this was only to show how to select a preferred provider for a
given package, you can now use \code{linux-ti-staging} again.

\section{BitBake tips}

BitBake is a powerful tool which can be used to execute specific commands. Here
is a list of some useful ones, used with the \code{virtual/kernel} package.

\begin{itemize}
  \item The Yocto recipes are divided into numerous tasks, you can print them
        by using: \code{bitbake -c listtasks virtual/kernel}.
  \item BitBake allows to call a specific task only (and its dependencies)
        with: \code{bitbake -c <task> virtual/kernel}. (\code{<task>} can be
        \code{menuconfig} here).
  \item You can force to rebuild a package by calling: \code{bitbake -f
        virtual/kernel}
  \item \code{world} is a special keyword for all packages. \code{bitbake
        --runall=fetch world} will download all packages sources (and their
        dependencies).
  \item You can get a list of locally available packages and their current
        version with: \\
        \code{bitbake -s}
  \item You can also find detailed information on available packages, their
        current version, dependencies or the contact information of the
        maintainer by visiting: \\
        \url{http://recipes.yoctoproject.org/}
\end{itemize}

For detailed information, please run \code{bitbake -h}

\section{Going further}

If you have some time left, let's improve our setup to use TFTP, in
order to avoid having to reflash the SD card for every test. What you
need to do is:

\begin{enumerate}

\item Install a TFTP server (package \code{tftpd-hpa}) on your system.

\item Copy the Linux kernel image and Device Tree to
  the TFTP server home directory (specified in
  \code{/etc/default/tftpd-hpa}) so that they are made available by the TFTP
  server.

\item Change the U-Boot \code{bootcmd} to load the kernel image and
  the Device Tree over TFTP.

\end{enumerate}

See the training materials of our {\em Embedded Linux system
  development} course for details!
