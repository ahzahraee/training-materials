\section{I/O Memory}

\begin{frame}
  \frametitle{Memory-Mapped I/O}
  \begin{columns}
  \column{0.75\textwidth}
  \begin{itemize}
    \item Same address bus to address memory and I/O device registers
    \item Access to the I/O device registers using regular instructions
    \item Most widely used I/O method across the different
      architectures supported by Linux
  \end{itemize}
  \column{0.25\textwidth}
    \includegraphics[width=\textwidth]{slides/kernel-driver-development-io-memory/mmio-vs-pio.pdf}
  \end{columns}
\end{frame}

\begin{frame}[fragile]
  \frametitle{Requesting I/O memory}
  \begin{itemize}
  \item Tells the kernel which driver is using which I/O registers
  \item
\begin{minted}{c}
struct resource *request_mem_region(unsigned long start,
                                    unsigned long len, char *name);
\end{minted}
  \item
\begin{minted}{c}
void release_mem_region(unsigned long start, unsigned long len);
  \end{minted}
  \item Allows to prevent other drivers requesting the same I/O registers,
    but is purely voluntary.
\end{itemize}
\end{frame}

\begin{frame}[fragile]
  \frametitle{/proc/iomem example - ARM 32 bit (BeagleBone Black, Linux 5.11)}
  \tiny
  \begin{columns}
  \column{0.5\textwidth}
  \begin{verbatim}
40300000-4030ffff : 40300000.sram sram@0
44e00c00-44e00cff : 44e00c00.prm prm@c00
44e00d00-44e00dff : 44e00d00.prm prm@d00
44e00e00-44e00eff : 44e00e00.prm prm@e00
44e00f00-44e00fff : 44e00f00.prm prm@f00
44e01000-44e010ff : 44e01000.prm prm@1000
44e01100-44e011ff : 44e01100.prm prm@1100
44e01200-44e012ff : 44e01200.prm prm@1200
44e07000-44e07fff : 44e07000.gpio gpio@0
44e09000-44e0901f : serial
44e0b000-44e0bfff : 44e0b000.i2c i2c@0
44e10800-44e10a37 : pinctrl-single
44e10f90-44e10fcf : 44e10f90.dma-router dma-router@f90
48024000-48024fff : 48024000.serial serial@0
48042000-480423ff : 48042000.timer timer@0
48044000-480443ff : 48044000.timer timer@0
\end{verbatim}
  \column{0.5\textwidth}
  \begin{verbatim}
48046000-480463ff : 48046000.timer timer@0
48048000-480483ff : 48048000.timer timer@0
4804a000-4804a3ff : 4804a000.timer timer@0
4804c000-4804cfff : 4804c000.gpio gpio@0
48060000-48060fff : 48060000.mmc mmc@0
4819c000-4819cfff : 4819c000.i2c i2c@0
481a8000-481a8fff : 481a8000.serial serial@0
481ac000-481acfff : 481ac000.gpio gpio@0
481ae000-481aefff : 481ae000.gpio gpio@0
481d8000-481d8fff : 481d8000.mmc mmc@0
49000000-4900ffff : 49000000.dma edma3_cc
4a100000-4a1007ff : 4a100000.ethernet ethernet@0
4a101200-4a1012ff : 4a100000.ethernet ethernet@0
80000000-9fdfffff : System RAM
80008000-80cfffff : Kernel code
80e00000-80f3d807 : Kernel data
\end{verbatim}
  \end{columns}
\end{frame}

\begin{frame}[fragile]
  \frametitle{Mapping I/O memory in virtual memory}
  \begin{itemize}
  \item Load/store instructions work with virtual addresses
  \item To access I/O memory, drivers need to have a virtual address
    that the processor can handle, because I/O memory is not mapped by
    default in virtual memory.
  \item The \code{ioremap} function satisfies this need:
\begin{minted}{c}
#include <asm/io.h>

void __iomem *ioremap(phys_addr_t phys_addr, unsigned long size);
void iounmap(void __iomem *addr);
\end{minted}
  \item Caution: check that \kfunc{ioremap} doesn't return a \code{NULL} address!
  \end{itemize}
\end{frame}

\begin{frame}[fragile]
  \frametitle{ioremap()}
  \begin{center}
    \includegraphics[height=0.75\textheight]{slides/kernel-driver-development-io-memory/ioremap.pdf}\\
    \code{ioremap(0xAFFEBC00, 4096) = 0xCDEFA000}
  \end{center}
\end{frame}

\begin{frame}[fragile]
  \frametitle{Managed API}
  Using \kfunc{request_mem_region} and \kfunc{ioremap} in device
  drivers is now deprecated. You should use the below "managed"
  functions instead, which simplify driver coding and error handling:
  \begin{itemize}
  \item \kfunc{devm_ioremap}, \kfunc{devm_iounmap}
  \item \kfunc{devm_ioremap_resource}
        \begin{itemize}
	\item Takes care of both the request and remapping operations!
	\end{itemize}
  \item \kfunc{devm_platform_ioremap_resource}
        \begin{itemize}
	\item Takes care of \kfunc{platform_get_resource},
	      \kfunc{request_mem_region} and \kfunc{ioremap}
	\item Caution: unlike the other \code{devm_} functions, its
	      first argument is of type \kstruct{pdev}, not a pointer to \kstruct{device}:
	\item Example: \kfile{drivers/char/hw_random/st-rng.c}:
	\begin{block}{}
	\begin{minted}{c}
base = devm_platform_ioremap_resource(pdev, 0);
if (IS_ERR(base))
        return PTR_ERR(base);
	\end{minted}
	\end{block}{}
	\end{itemize}
  \end{itemize}
\end{frame}

\begin{frame}[fragile]
  \frametitle{Accessing MMIO devices}
  \begin{itemize}
  \item Directly reading from or writing to addresses returned by
    \kfunc{ioremap} (\emph{pointer dereferencing}) may not work on some
    architectures.
  \item To do PCI-style, little-endian accesses (byte swapping being done
    automatically assuming a little-endian device):
\begin{minted}{c}
unsigned read[bwlq](void *addr);
void write[bwlq](unsigned val, void *addr);
\end{minted}
  \item To do raw access, without endianness conversion
\begin{minted}{c}
unsigned __raw_read[bwlq](void *addr);
void __raw_write[bwlq](unsigned val, void *addr);
\end{minted}
  \item Little-endian is more frequent and also easier
        to use in drivers. Even if you just read the least significant
        byte of a 32-bit register, it's still at the same address.
  \item Example
    \begin{itemize}
    \item 32 bit write (\kfile{drivers/tty/serial/uartlite.c}):
\begin{minted}{c}
writel(c & 0xff, port->membase + 4);
\end{minted}
    \end{itemize}
  \end{itemize}
\end{frame}

\begin{frame}
  \frametitle{Avoiding I/O access issues}
  \begin{itemize}
  \item Caching on I/O memory already disabled
  \item Use the \kfunc{writel}/\kfunc{readl} macros,
        they do the right thing for your architecture
  \item The compiler and/or CPU can reorder memory accesses, which
    might cause trouble for your devices is they expect one register
    to be read/written before another one.
    \begin{itemize}
    \item Memory barriers are available to prevent this reordering
    \item \kfunc{rmb} is a read memory barrier, prevents reads to
      cross the barrier
    \item \kfunc{wmb} is a write memory barrier
    \item \kfunc{mb} is a read-write memory barrier
    \item Starts to be a problem with CPUs that reorder instructions and
          with SMP. See \kdoctext{memory-barriers.txt} for details.
    \end{itemize}
  \item Note that \kfunc{readl}, \kfunc{writel} and similar functions
        already contain barriers (safer), while the raw ones don't.
  \end{itemize}
\end{frame}

\begin{frame}
  \frametitle{/dev/mem}
  \begin{itemize}
  \item Used to provide user space applications with direct access to
    physical addresses.
  \item Usage: open \code{/dev/mem} and read or write at given offset.
    What you read or write is the value at the corresponding physical
    address.
  \item Used by applications such as the X server to write directly to
    device memory.
  \item On \code{x86}, \code{arm}, \code{arm64}, \code{riscv},
    \code{powerpc}, \code{parisc}, \code{s390}:
    \kconfig{CONFIG_STRICT_DEVMEM} option to restrict \code{/dev/mem}
    to non-RAM addresses, for security reasons (Linux 5.12 status).
    \kconfig{CONFIG_IO_STRICT_DEVMEM} goes beyond and only allows to access
    {\em idle} I/O ranges (not appearing in \code{/proc/iomem}).
\end{itemize}
\end{frame}
