\section{Licensing}

\subsection{Managing licenses}

\begin{frame}[fragile]
  \frametitle{Tracking license changes}
  \begin{itemize}
    \item The license of an external project may change at some point.
    \item The \code{LIC_FILES_CHKSUM} tracks changes in the license
      files.
    \item If the license's checksum changes, the build will fail.
      \begin{itemize}
        \item The recipe needs to be updated.
      \end{itemize}
    \begin{block}{}
    \begin{minted}{c}
LIC_FILES_CHKSUM = "                                \
    file://COPYING;md5=...                          \
    file://src/file.c;beginline=3;endline=21;md5=..."
    \end{minted}
    \end{block}
    \item \code{LIC_FILES_CHKSUM} is mandatory in every recipe, unless
      \code{LICENSE} is set to \code{CLOSED}.
  \end{itemize}
\end{frame}

\begin{frame}[fragile]
  \frametitle{Package exclusion}
  \begin{itemize}
    \item We may not want some packages due to their licenses.
    \item To exclude a specific license, use
      \code{INCOMPATIBLE_LICENSE}
      \item To exclude all GPLv3 packages:
      \begin{block}{}
      \begin{minted}{c}
INCOMPATIBLE_LICENSE = "GPLv3"
      \end{minted}
      \end{block}
    \item License names are the ones used in the \code{LICENSE}
      variable.
    \item The \code{meta-gplv2} layer provides recipes for software
      where upstream has moved to GPLv3 licenses.
  \end{itemize}
\end{frame}

\begin{frame}[fragile]
  \frametitle{Commercial licenses}
  \begin{itemize}
    \item By default the build system does not include commercial
      components.
    \item Packages with a commercial component define:
      \begin{block}{}
      \begin{minted}{c}
LICENSE_FLAGS = "commercial"
      \end{minted}
      \end{block}
    \item To build a package with a commercial component, the package
      must be in the \code{LICENSE_FLAGS_WHITELIST} variable.
    \item Example, \code{gst-plugins-ugly}:
      \begin{block}{}
      \begin{minted}[fontsize=\small]{c}
LICENSE_FLAGS_WHITELIST = "commercial_gst-plugins-ugly"
      \end{minted}
      \end{block}
  \end{itemize}
\end{frame}

\begin{frame}[fragile]
  \frametitle{Listing licenses}
  OpenEmbbedded will generate a manifest of all the licenses of the
  software present on the target image in
  \code{$BUILDDIR/tmp/deploy/licenses/<image>/license.manifest}
  \begin{block}{}
    \fontsize{9}{9}\selectfont
    \begin{minted}{console}
PACKAGE NAME: busybox
PACKAGE VERSION: 1.31.1
RECIPE NAME: busybox
LICENSE: GPLv2 & bzip2-1.0.6

PACKAGE NAME: dropbear
PACKAGE VERSION: 2019.78
RECIPE NAME: dropbear
LICENSE: MIT & BSD-3-Clause & BSD-2-Clause & PD
    \end{minted}
  \end{block}
\end{frame}

\begin{frame}[fragile]
  \frametitle{Providing license text}
  To include the license text in the root filesystem either:
  \begin{itemize}
  \item Use \code{COPY_LIC_DIRS = "1"} and \code{COPY_LIC_MANIFEST =
    "1"}
  \item or use \code{LICENSE_CREATE_PACKAGE = "1"} to generate
    packages including the license and install the required license
    packages.
  \end{itemize}
\end{frame}

\begin{frame}[fragile]
  \frametitle{Providing sources}
  OpenEmbbedded provides the \code{archiver} class to generate
  tarballs of the source code:
  \begin{itemize}
  \item Use \code{INHERIT += "archiver"}
  \item Set the \code{ARCHIVER_MODE} variable, the default is to
    provide patched sources. To provide configured sources:
  \end{itemize}
    \begin{block}{}
      \fontsize{9}{9}\selectfont
      \begin{minted}{bash}
ARCHIVER_MODE[src] = "configured"
      \end{minted}
    \end{block}
\end{frame}
