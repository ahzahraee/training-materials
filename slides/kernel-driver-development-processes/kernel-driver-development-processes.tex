\section{Processes, scheduling and interrupts}

\subsection{Processes and scheduling}

\begin{frame}
  \frametitle{Process, thread?}
  \begin{itemize}
  \item Confusion about the terms \emph{process}, \emph{thread} and
    \emph{task}
  \item In UNIX, a process is created using \code{fork()} and is
    composed of
    \begin{itemize}
    \item An address space, which contains the program code, data,
      stack, shared libraries, etc.
    \item A single thread, which is the only entity known by the scheduler.
    \end{itemize}
  \item Additional threads can be created inside an existing process,
    using \code{pthread_create()}
    \begin{itemize}
    \item They run in the same address space as the initial thread of
      the process
    \item They start executing a function passed as argument to
      \code{pthread_create()}
    \end{itemize}
  \end{itemize}
\end{frame}

\begin{frame}
  \frametitle{Process, thread: kernel point of view}
  \begin{itemize}
  \item In kernel space, each thread running in the system is
    represented by a structure of type \kstruct{task_struct}
  \item From a scheduling point of view, it makes no difference
    between the initial thread of a process and all additional threads
    created dynamically using \code{pthread_create()}
  \end{itemize}
  \begin{center}
    \includegraphics[height=0.4\textheight]{slides/kernel-driver-development-processes/address-space.pdf}
  \end{center}
\end{frame}

\begin{frame}
  \frametitle{Relation between execution mode, address space and context}
  \begin{itemize}
  \item When speaking about \emph{process} and \emph{thread}, these
    concepts need to be clarified:
    \begin{itemize}
    \item \emph{Mode} is the level of privilege allowing to perform
      some operations:
      \begin{itemize}
      \item \emph{Kernel Mode}: in this level CPU can perform any
        operation allowed by its architecture; any instruction, any
        I/O operation, any area of memory accessed.
      \item \emph{User Mode}: in this level, certain instructions are
        not permitted (especially those that could alter the global
        state of the machine), some memory areas cannot be accessed.
      \end{itemize}
    \item Linux splits its \emph{address space} in \emph{kernel space}
      and \emph{user space}
      \begin{itemize}
      \item \emph{Kernel space} is reserved for code running in
        \emph{Kernel Mode}.
      \item \emph{User space} is the place were applications execute
        (accessible from \emph{Kernel Mode}).
      \end{itemize}
    \item \emph{Context} represents the current state of an execution flow.
      \begin{itemize}
      \item The \emph{process context} can be seen as the content of
        the registers associated to this process: execution register,
        stack register...
      \item The \emph{interrupt context} replaces the \emph{process
          context} when the interrupt handler is executed.
      \end{itemize}
    \end{itemize}
  \end{itemize}
\end{frame}

\begin{frame}
  \frametitle{A thread life}
  \begin{center}
    \includegraphics[height=0.8\textheight]{slides/kernel-driver-development-processes/threads-life.pdf}
  \end{center}
\end{frame}

\begin{frame}
  \frametitle{Execution of system calls}
  \begin{center}
    \includegraphics[width=\textwidth]{slides/kernel-driver-development-processes/syscalls.pdf}\\
    The execution of system calls takes place in the context of the
    thread requesting them.
  \end{center}
\end{frame}

