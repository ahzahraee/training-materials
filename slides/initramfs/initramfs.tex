\begin{frame}
  \frametitle{Root filesystem in memory: initramfs}
  It is also possible to boot the system with a filesystem in memory:
  {\em initramfs}
  \begin{itemize}
  \item Either from a compressed CPIO archive integrated into the kernel image
  \item Or from such an archive loaded by the bootloader into memory
  \item At boot time, this archive is extracted into the Linux file cache
  \item It is useful for two cases:
    \begin{itemize}
    \item Fast booting of very small root filesystems. As the
      filesystem is completely loaded at boot time, application
      startup is very fast.
    \item As an intermediate step before switching to a real root
      filesystem, located on devices for which drivers not part of the
      kernel image are needed (storage drivers, filesystem drivers,
      network drivers). This is always used on the kernel of
      desktop/server distributions to keep the kernel image size
      reasonable.
    \end{itemize}
  \item Details (in kernel sources): \\
    \kdochtml{filesystems/ramfs-rootfs-initramfs}\\
  \end{itemize}
\end{frame}

\begin{frame}[fragile]
  \frametitle{External initramfs}
  \begin{itemize}
  \item To create one, first create a compressed CPIO archive:
 \begin{verbatim}
cd rootfs/
find . | cpio -H newc -o > ../initramfs.cpio
cd ..
gzip initramfs.cpio
\end{verbatim}
  \item If you're using U-Boot, you'll need to include your archive
  in a U-Boot container:
  \begin{verbatim}
mkimage -n 'Ramdisk Image' -A arm -O linux -T ramdisk -C gzip \
        -d initramfs.cpio.gz uInitramfs
\end{verbatim}
  \item Then, in the bootloader, load the kernel binary, DTB and
  \code{uInitramfs} in RAM and boot the kernel as follows:
  \begin{verbatim}
bootz kernel-addr initramfs-addr dtb-addr
\end{verbatim}
  \end{itemize}
\end{frame}

\begin{frame}
  \frametitle{Built-in initramfs}
  \begin{columns}
  \column{0.75\textwidth}
  To have the kernel Makefile include an initramfs archive in
  the kernel image: use the \kconfig{CONFIG_INITRAMFS_SOURCE}
  option.
  \begin{itemize}
  \item It can be the path to a directory containing the root
    filesystem contents
  \item It can be the path to a ready made cpio archive
  \item It can be a text file describing the contents of the initramfs
  \end{itemize}
  See the kernel documentation for details:
  \kdochtml{driver-api/early-userspace/early_userspace_support}
  \column{0.25\textwidth}
    \includegraphics[width=0.9\textwidth]{slides/initramfs/initramfs.pdf}
  \end{columns}
\end{frame}
