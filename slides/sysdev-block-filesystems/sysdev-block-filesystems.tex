\section{Block filesystems}

\subsection{Block devices}

\begin{frame}
  \frametitle{Block vs. flash}
  \begin{itemize}
  \item Storage devices are classified in two main types: {\bf block
      devices} and {\bf flash devices}
    \begin{itemize}
    \item They are handled by different subsystems and different
      filesystems
    \end{itemize}
  \item {\bf Block devices} can be read and written to on a per-block
    basis, in random order, without erasing.
    \begin{itemize}
    \item Hard disks, RAM disks
    \item USB keys, SSD, SD cards, eMMC: these are based
      on flash storage, but have an integrated controller that
      emulates a block device, managing the flash in a transparent
      way.
    \end{itemize}
  \item {\bf Raw flash devices} are driven by a controller on the
      SoC. They can be read, but writing requires prior erasing,
      and often occurs on a larger size than the “block” size.
    \begin{itemize}
    \item NOR flash, NAND flash
    \end{itemize}
  \end{itemize}
\end{frame}

\begin{frame}[fragile]
  \frametitle{Block device list}
  \begin{itemize}
  \item The list of all block devices available in the system can be
    found in \code{/proc/partitions}\\
\begin{verbatim}
$ cat /proc/partitions
major minor #blocks name

 179        0    3866624 mmcblk0
 179        1      73712 mmcblk0p1
 179        2    3792896 mmcblk0p2
   8        0  976762584 sda
   8        1    1060258 sda1
   8        2  975699742 sda2
\end{verbatim}
  \item \code{/sys/block/} also stores information about each block device,
     for example whether it is removable storage or not.
  \end{itemize}
\end{frame}

\begin{frame}{Partitioning}
  \begin{itemize}
  \item Block devices can be partitioned to store different parts of a
    system
  \item The partition table is stored inside the device itself, and is
    read and analyzed automatically by the Linux kernel
    \begin{itemize}
    \item \code{mmcblk0} is the entire device
    \item \code{mmcblk0p2} is the second partition of \code{mmcblk0}
    \end{itemize}
  \item Two partition table formats:
    \begin{itemize}
    \item {\em MBR}, the legacy format
    \item {\em GPT}, the new format, not used everywhere yet, supporting
      disks bigger than 2 TB.
    \end{itemize}
  \item Numerous tools to create and modify the partitions on a block
    device: \code{fdisk}, \code{cfdisk}, \code{sfdisk}, \code{parted},
    etc.
  \end{itemize}
\end{frame}

\begin{frame}{Transfering data to a block device}
  \begin{itemize}
  \item It is often necessary to transfer data to or from a block
    device in a {\em raw} way
    \begin{itemize}
    \item Especially to write a {\em filesystem image} to a block
      device
    \end{itemize}
  \item This directly writes to the block device itself, bypassing any
    filesystem layer.
  \item The block devices in \code{/dev/} allow such {\em raw} access
  \item \code{dd} ({\em {\bf d}isk {\bf d}uplicate})
    is the tool of choice for such transfers:
    \begin{itemize}
    \item \code{dd if=/dev/mmcblk0p1 of=testfile bs=1M count=16}\\
      Transfers 16 blocks of 1 MB from \code{/dev/mmcblk0p1} to
      \code{testfile}
    \item \code{dd if=testfile of=/dev/sda2 bs=1M seek=4}\\
      Transfers the complete contents of \code{testfile} to
      \code{/dev/sda2}, by blocks of 1 MB, but starting at offset 4 MB
      in \code{/dev/sda2}
    \item {\bf Typical mistake}: copying a file to a filesystem without
      mounting it first:\\
      \code{dd if=zImage of=/dev/sde1}\\
      Instead, you should use:\\
      \code{sudo mount /dev/sde1 /boot}\\
      \code{cp zImage /boot/}\\
    \end{itemize}
  \end{itemize}
\end{frame}

\subsection{Available block filesystems}

\begin{frame}{Standard Linux filesystem format: ext2, ext3, ext4}
  \begin{itemize}
  \item The standard filesystem used on Linux systems is the series of
    \code{ext{2,3,4}} filesystems
    \begin{itemize}
    \item \code{ext2}
    \item \code{ext3}, brought {\em journaling} compared to \code{ext2},
	  now obsoleted by \code{ext4}.
    \item \code{ext4}, mainly brought performance improvements and
          support for very big partitions.
    \end{itemize}
  \item It supports all features Linux needs in a root filesystem:
    permissions, ownership, device files, symbolic links, etc.
  \end{itemize}
\end{frame}

\begin{frame}
  \frametitle{Journaled filesystems}
  \begin{columns}
    \column{0.6\textwidth}
    \begin{itemize}
    \item Unlike simpler filesystems (\code{ext2}, \code{vfat}...),
      designed to stay in a coherent state even after system
      crashes or a sudden poweroff.
    \item Writes are first described in the journal before being
      committed to files (can be all writes, or only metadata writes
      depending on the configuration)
    \item Allows to skip a full disk check at boot time after an
      unclean shutdown.
    \end{itemize}
    \column{0.4\textwidth}
    \includegraphics[width=\textwidth]{slides/sysdev-block-filesystems/journal.pdf}
  \end{columns}
\end{frame}

\begin{frame}
  \frametitle{Filesystem recovery after crashes}
  \begin{columns}
    \column{0.4\textwidth}
    \includegraphics[width=\textwidth]{slides/sysdev-block-filesystems/journal-recovery.pdf}
    \column{0.6\textwidth}
    \begin{itemize}
    \item Thanks to the journal, the recovery at boot time is quick,
      since the operations in progress at the moment of the unclean
      shutdown are clearly identified. There's no need for a full
      filesystem check.
    \item Does not mean that the latest writes made it to the storage:
      this depends on syncing the changes to the filesystem.
    \end{itemize}
    See \url{https://en.wikipedia.org/wiki/Journaling_file_system}
    for further details.
  \end{columns}
\end{frame}

\begin{frame}
  \frametitle{Other journaled Linux/UNIX filesystems}
  \begin{itemize}
  \item \code{btrfs}, intended to become the next standard filesystem
    for Linux. Integrates numerous features: data checksuming,
    integrated volume management, snapshots, etc.
  \item \code{XFS}, high-performance filesystem inherited from SGI
    IRIX, still actively developed.
  \item \code{JFS}, inherited from IBM AIX. No longer actively
    developed, provided mainly for compatibility.
  \item \code{reiserFS}, used to be a popular filesystem, but its
    latest version \code{Reiser4} was never merged upstream.
  \item \code{ZFS}, provides standard and advanced filesystem and
    volume management (CoW, snapshot, etc.). Due to license it
    cannot be mainlined into Linux but present into few distributions
    (see \code{OpenZFS}).
  \end{itemize}
  All those filesystems provide the necessary functionalities for
  Linux systems: symbolic links, permissions, ownership, device files,
  etc.
\end{frame}

\begin{frame}
  \frametitle{F2FS: filesystem for flash-based storage}
  \url{https://en.wikipedia.org/wiki/F2FS}
  \begin{itemize}
  \item Filesystem that takes into account the characteristics of
    flash-based storage: eMMC, SD cards, SSD, etc.
  \item Developed and contributed by Samsung
  \item Now supporting transparent compression (LZO, LZ4, zstd) and
        encryption.
  \item For optimal results, need a number of details about the
    storage internal behavior which may not easy to get
  \item Benchmarks: best performer on flash devices most of the time: \\
        See \url{https://lwn.net/Articles/520003/}
  \item Technical details: \url{https://lwn.net/Articles/518988/}
  \item Not as widely used as \code{ext4} and \code{btrfs}, even on flash-based
    storage.
  \end{itemize}
\end{frame}

\begin{frame}
  \frametitle{SquashFS: read-only filesystem}
  \begin{itemize}
  \item Read-only, compressed filesystem for block devices. Fine for
    parts of a filesystem which can be read-only (kernel, binaries...)
  \item Great compression rate, which generally brings improved read
    performance
  \item Used in most live CDs and live USB distributions
  \item Supports several compression algorithms (LZO, XZ, etc.)
  \item Benchmarks: roughly 3 times smaller than ext3, and 2-4 times
    faster (\url{https://elinux.org/Squash_Fs_Comparisons})
  \item New alternative to SquashFS: EROFS\\
    \url{https://en.wikipedia.org/wiki/EROFS}
  \end{itemize}
\end{frame}

\begin{frame}
  \frametitle{Our advice for choosing the best filesystem}
  \begin{itemize}
     \item Some filesystems will work better than others
           depending on how you use them.
     \item For example, \code{reiserFS} had the reputation
           to be best at handling many small files.
     \item Fortunately, filesystems are easy to benchmark, being
           transparent to applications:
           \begin{itemize}
             \item Format your storage with each filesystem
	     \item Copy your data to it
	     \item Run your system on it and benchmark its
                   performance.
	     \item Keep the one working best in your case.
           \end{itemize}
     \item For read/write partitions, the best choices are
           probably \code{btrfs} and \code{f2fs}.
  \end{itemize}
\end{frame}

\begin{frame}
  \frametitle{Compatibility filesystems}
  Linux also supports several other filesystem formats, mainly to be
  interoperable with other operating systems:
  \begin{itemize}
  \item \code{vfat} for compatibility with the FAT filesystem used in
    the Windows world and on numerous removable devices
    \begin{itemize}
    \item Also convenient to store bootloader binaries (FAT easy
      to understand for ROM code)
    \item This filesystem does {\em not} support features like
      permissions, ownership, symbolic links, etc. Cannot be used for
      a Linux root filesystem.
    \item Linux now supports the exFAT filesystem too (\code{exfat}).
    \end{itemize}
  \item \code{ntfs} for compatibility with the NTFS filesystem used on
    Windows
  \item \code{hfs} for compatibility with the HFS filesystem used on
    Mac OS
  \item \code{iso9660}, the filesystem format used on CD-ROMs,
    obviously a read-only filesystem
  \end{itemize}
\end{frame}

\begin{frame}
  \frametitle{tmpfs: filesystem in RAM}
  \begin{itemize}
  \item Not a block filesystem of course!
  \item Perfect to store temporary data in RAM: system log files,
    connection data, temporary files...
  \item More space-efficient than ramdisks: files are directly in the
    file cache, grows and shrinks to accommodate stored files
  \item How to use: choose a name to distinguish the various tmpfs
    instances you have (unlike in most other filesystems, each
    tmpfs instance is different). Examples:\\
    \code{mount -t tmpfs run /var/run}\\
    \code{mount -t tmpfs shm /dev/shm}
  \item See \kdochtml{filesystems/tmpfs} in kernel sources.
  \end{itemize}
\end{frame}

\subsection{Using block filesystems}

\begin{frame}
  \frametitle{Creating ext2/ext4 filesystems}
  \begin{itemize}
  \item To create an empty ext2/ext4 filesystem on a block device or
    inside an already-existing image file
    \begin{itemize}
    \item \code{mkfs.ext2 /dev/hda3}
    \item \code{mkfs.ext4 /dev/sda3}
    \item \code{mkfs.ext2 disk.img}
    \end{itemize}
  \item To create a filesystem image from a directory containing all
    your files and directories
    \begin{itemize}
    \item Use the \code{genext2fs} tool, from the package of the same name
    \item This tool only supports ext2. Alternative for other
          filesystems: create a disk image, format it, mount it (see
          next slides), copy contents and umount.
    \item \code{genext2fs -d rootfs/ rootfs.img}
    \item Your image is then ready to be transferred to your block
      device
    \end{itemize}
  \end{itemize}
\end{frame}

\begin{frame}
  \frametitle{Mounting filesystem images}
  \begin{itemize}
  \item Once a filesystem image has been created, one can access and
    modifies its contents from the development workstation, using the
    {\bf loop} mechanism
  \item Example:\\
    \code{genext2fs -d rootfs/ rootfs.img}\\
    \code{mkdir /tmp/tst}\\
    \code{mount -t ext2 -o loop rootfs.img /tmp/tst}
  \item In the \code{/tmp/tst} directory, one can access and modify
    the contents of the \code{rootfs.img} file.
  \item This is possible thanks to \code{loop}, which is a kernel
    driver that emulates a block device with the contents of a file.
  \item Note: \code{-o loop} no longer necessary with recent versions
        of \code{mount} from {\em GNU Coreutils}. Not true with BusyBox
       \code{mount}.
  \item Do not forget to run \code{umount} before using the filesystem
    image!
  \end{itemize}
\end{frame}

\begin{frame}
  \frametitle{Creating squashfs filesystems}
  \begin{itemize}
  \item Need to install the \code{squashfs-tools} package
  \item Can only create an image: creating an empty {\em squashfs}
    filesystem would be useless, since it's read-only.
  \item To create a {\em squashfs} image:
    \begin{itemize}
    \item \code{mksquashfs rootfs/ rootfs.sqfs -noappend}
    \item \code{-noappend}: re-create the image from scratch rather
      than appending to it
    \end{itemize}
  \item Examples mounting a squashfs filesystem:
    \begin{itemize}
    \item Same way as for other block filesystems
    \item \code{mount -o loop rootfs.sqfs /mnt} (filesystem image on the host)
    \item \code{mount /dev/<device> /mnt} (on the target)
    \end{itemize}
  \end{itemize}
\end{frame}

\begin{frame}
  \frametitle{Mixing read-only and read-write filesystems}
  \begin{columns}
    \column{0.7\textwidth}
    Good idea to split your block storage into:
    \begin{itemize}
    \item A compressed read-only partition (\code{SquashFS})\\
      Typically used for the root filesystem (binaries, kernel...).\\
      Compression saves space. Read-only access protects your system
      from mistakes and data corruption.
    \item A read-write partition with a journaled filesystem (like \code{ext4})\\
      Used to store user or configuration data.\\
      Guarantees filesystem integrity after power off or crashes.
    \item Ram storage for temporary files (\code{tmpfs})
    \end{itemize}
    \column{0.3\textwidth}
    \includegraphics[width=\textwidth]{slides/sysdev-block-filesystems/mixing-filesystems.pdf}
  \end{columns}
\end{frame}

\begin{frame}
  \frametitle{Issues with flash-based block storage}
  \begin{itemize}
  \item Flash storage made available only through a block interface.
  \item Hence, no way to access a low level flash interface
    and use the Linux filesystems doing wear leveling.
  \item No details about the layer (Flash Translation Layer) they
    use. Details are kept as trade secrets, and may hide poor
    implementations.
  \item Not knowing about the wear leveling algorithm, it is highly
    recommended to limit the number of writes to these devices.
  \item Using industrial grade storage devices (MMC/SD, USB) is
    also recommended.
  \end{itemize}
  See the \href{https://lwn.net/Articles/428584/}{\em Optimizing Linux with
  cheap flash drives} article from Arnd Bergmann and try his {\em
  flashbench} tool (\url{http://git.linaro.org/people/arnd/flashbench.git/about/})
  for finding out the erase block and page size for your storage, and
  optimizing your partitions and filesystems for best performance.
\end{frame}

\setuplabframe
{Block filesystems}
{
  \begin{itemize}
  \item Creating partitions on your block storage
  \item Booting your system with a mix of filesystems: SquashFS for
    the root filesystem (including applications), ext4 for
    configuration and user data, and tmpfs for
    temporary system files.
  \end{itemize}
}
